\chapter{KẾT LUẬN VÀ HƯỚNG PHÁT TRIỂN} \label{chapter:Conclusion}

Chúng tôi đã cải tiến thuật toán ACO++ bằng cách tích hợp hai cơ chế kiểm soát tham số, kỹ thuật Bay hơi lười biếng và kỹ thuật Phân cụm thứ bậc để tăng cường tính linh hoạt và hiệu suất. Hầu hết các tham số của chúng tôi được điều chỉnh tự động, một số trở nên tự thích ứng nhờ vào ES và một số khác trở nên thích ứng thông qua cơ chế thích ứng lấy cảm hứng từ AACO-NC. Kiến di chuyển trên các cạnh của cây cụm thứ bậc thay vì di chuyển từ thành phố này sang thành phố khác. Phân cụm thứ bậc đã giúp làm giảm chi phí tìm kiếm tuyến đường. Bay hơi lười biếng giúp giảm thiểu độ phức tạp thời gian của giai đoạn bay hơi bằng cách theo dõi trạng thái mong muốn và lịch sử, và chỉ bay hơi pheromone trên các cạnh được sử dụng. Trong khi ACO++ yêu cầu điều chỉnh riêng cấu hình siêu tham số cho mỗi nhóm trường hợp, SAAS mang lại kết quả xuất sắc với chỉ một cấu hình chạy trên tất cả 432 trường hợp đánh giá.

Trong công trình này, chúng tôi chỉ mới tập trung cải tiến ở giai đoạn tìm đường và chất lượng của tham số. Ở bài toán Điều Hướng Thu Thập và các thuật toán trước đó, tồn tại hai giai đoạn quan trọng khác mà chúng tôi chưa chạm đến. Đó là xây dựng chiến lược thu thập và tìm kiếm cục bộ. Các thuật toán cho bài toán Người Thu Thập Du Lịch (TTP) sở hữu các phép tìm kiếm cục bộ rất truyền cảm hứng \cite{Namazi2023SolvingTT}. Ta có thể điều chỉnh chúng để áp dụng được cho ThOP. Chiến lược thu thập của SAAS, được kế thừa từ ACO++, có độ phức tạp cao nhất trong tất cả giai đoạn, chứa nhiều khả năng có thể cải tiến.

Bộ trường hợp đánh giá của ThOP rất đa dạng và bao phủ rộng các khả năng. Tuy vậy, nó vẫn tồn tại vài thiếu sót không giống với thực tế. Ví dụ như mỗi thành phố đều có số lượng vật phẩm như nhau và bản đồ mang tính đối xứng. Bản đồ đối xứng nghĩa là quãng được từ A đến B và từ B đến A là như nhau. Điều này có thể dẫn đến đánh giá sai tiềm năng của thuật toán trong bối cảnh thực tế. Các nghiên cứu tương lai có thể hướng đến cải thiện yếu điểm này.

Các kĩ thuật chúng tôi đề xuất có khả năng tích hợp vào các phương pháp tân tiến trên nhiều bài toán khác, từ đó giúp cải thiện chất lượng lời giải. Ta cũng có thể tích hợp các yếu tố cửa sổ thời gian (time window), nhiều điểm kho chứa (multiple depot) hoặc đa mục tiêu vào bài toán Điều Hướng Thu Thập để thúc đẩy bài toán nghiên cứu gần hơn với vấn đề thực tế. 
