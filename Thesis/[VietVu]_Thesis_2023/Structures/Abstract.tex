\begin{abstract}
\addchaptertocentry{\titleAbstract}

Bài toán Điều Hướng Thu Thập (Thief Orienteering Problem) là một vấn đề đa thành phần với hai bài toán con tương tác là bài toán Ba Lô (Knapsack Problem) và bài toán Điều Hướng (Orienteering Problem). ACO++, một metaheuristic hiện đại cho ThOP, kết hợp thuật toán MAX-MIN Ant System để xây dựng đường đi, một thuật toán ngẫu nhiên cho việc tạo chiến lược thu thập, và phương pháp 2-OPT cho tìm kiếm cục bộ (local search). Tuy nhiên, hiệu suất xuất sắc được báo cáo của ACO++ được đạt được bằng cách sử dụng các bộ giá trị tham số khác nhau đã được điều chỉnh tỉ mỉ cho từng nhóm cụ thể của các trường hợp đánh giá. Trong công trình này, chúng tôi đề xuất một biến thể tự thích ứng mới của ACO++. Không đòi hỏi quá trình điều chỉnh phức tạp, phương pháp của chúng tôi sử dụng các cơ chế thích ứng để điều chỉnh các tham số cho từng trường hợp vấn đề cụ thể trong quá trình chạy thuật toán. Chúng tôi cũng sử dụng kỹ thuật bay hơi lười biếng và tận dụng phân cụm thứ bậc để cải thiện hiệu suất của đàn kiến trong việc khám phá không gian tìm kiếm. Trong số 432 trường hợp đánh giá, phương pháp Self-Adaptive Ant System (SAAS) của chúng tôi tạo ra kết quả vượt trội hơn so với các phương pháp tân tiến trước đó. Mã nguồn đã được công bố tại \url{https://github.com/ELO-Lab/SAAS-HC}.

\end{abstract}